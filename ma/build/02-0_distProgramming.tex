\section{Distributed Programming}
In 1937 a paper titled \textit{``On computable numbers, with an
application to the Entscheidungsproblem''}, written by Alan Turing, was
published \cite{turingcomputable}. In it Turing defines what is now
known as a \textit{Turing machine}, a theoretical machine model for
executing arbitrary computation. He demonstrates how to formulate
instructions for this machine in order to configure its execution
behavior and goes so far as to show how this
machine could even receive another machine configuration as its input
and then execute what the other machine would have done, thereby
effectively emulating the given machine.
\newline

But he wasn't the only one thinking about computation. As summarized
by Denning in \cite{whatiscomputation}, Kurt Gödel, Alonzo Church and Emil
Post all contributed to the task of exploring the boundaries of
computation, setting the tone for the next 80 years of research
concerning \textit{computation}, \textit{computers} and \textit{automation}.

Of all the proposed approaches of how to tackle the concept of computation
and turning it into a tool that could be understood and used by delivering
a theoretical framework, one could argue that Turing's machine model
emerged as a winner because, although not at the time, it turned out
to be closest to what machines became to be. How they were structed and
how they basically operated. Still, to this day.

So in order to use or \textit{program} these machines, these computers,
programming languages were developed and over the decades have gone
through their own evolutionary process \cite{pl-gens} with the
first generation using binary machine instructions to todays languages
that allow for higher level programming styles like
\textit{object oriented programming } \cite{bjarneOO} or
\textit{functional programming} \cite{wadler-functional}.
\newline

But no matter what mainstream programming language one looks at today,
they all seem to have one thing in common, which also gets replicated
by each new language that arrives: they are all focussing on
describing computation. Computation executed by a single core machine.
Because for most of the last century, this was the problem we were
facing.

However, with the advent of commodity multicore hardware
\cite{core2duo} the problem domain changed. Now a machine isn't
a single machine anymore, but rather a group of multiple CPU cores
that can operate and compute independent of each other.

Unfortunately mainstream languages are still struggling with delivering
language concepts and semantics that allow for utilizing this new
hardware. In Fig.\ref{pythreads} one can see a very minimal multithreaded
program written in Python, one of the most used languages today
\cite{langrank}.


\begin{wrapfigure}{l}{0.55\textwidth}
  \vspace{-10pt}
  \begin{center}
    \begin{lstlisting}
from threading import Thread

def count(n):
  while n > 0:
    n -= 1

t1 = Thread(target=count,args=(1000000,))
t1.start()

t2 = Thread(target=count,args=(1000000,))
t2.start()

t1.join()
t2.join()
    \end{lstlisting}
  \end{center}
  \vspace{-20pt}
  \caption{Python multithreading example}
  \label{pythreads}
\end{wrapfigure}

As was shown by Beazley in \cite{pygil} this multithreaded version is up to
\textit{two times} slower than the single threaded version and the
reason for this is the so called \textit{Global Interpreter Lock}
(GIL) used by the Python interpreter which basically prevents multiple
native threads from truly executing in parallel.

This shows how important it is to reevaluate exisiting programming
languages and methodologies whenever the underlying machine or system
changes dramatically.
\newline

Another such paradigm shift was the advent of
the \textit{internet}, or large scale networking in general, because it
dawned what is now known as the \textit{information age}. Although this
is a rather broad term, describing global change in almost all areas of
modern civilization, I believe these changes can also be seen in the area
of computing and computation. Because in order to deliver products
like: the world wide web, internet search, internet advertising,
social networks, social media, music streaming,
video streaming, messenger services, navigation/maps, e-commerce etc.
companies had to build massive computer networks in order to either store
massive amounts of data or to simply scale their
product to millions or billions of users.
And altough almost all of the current products, apps, or services
could not exist, without the computation that is implemented inside
their core parts, it is \textit{communication} that has become the
main obstacle for building large scale \textit{information systems}.
