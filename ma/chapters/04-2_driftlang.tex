\subsection{The Drift Language}
\label{driftlang}

And so, in that same spirit, the basic concepts of the \textit{Drift}
language are \textit{names} to represent data and \textit{services}
to represent functionality.

\begin{figure}[h]
    \centering
    \begin{subfigure}[b]{0.4\textwidth}

  \begin{lstlisting}
.> Cat mydata
Lorem ipsum dolor sit amet,
consectetur adipiscing elit.
.>
  \end{lstlisting}

        \caption{Simple service invocation.}
        \label{cat1}
    \end{subfigure}
    \hspace{20pt} %add desired spacing between images, e. g. ~, \quad, \qquad, \hfill etc.
      %(or a blank line to force the subfigure onto a new line)
    \begin{subfigure}[b]{0.4\textwidth}

  \begin{lstlisting}
.> myresult = Cat mydata
.>
  \end{lstlisting}

        \caption{Service invocation with result bound to a name.}
        \label{cat2}
    \end{subfigure}
    \caption{Basic service invocation mechanism in Drift.}\label{drift-examples1}
\end{figure}

Fig.\ref{drift-examples1} shows the basic concepts offered by
the \textit{Drift language} as implemented by the \textit{Drift shell}.
In the Drift language, service names are required to
start with an upper case letter, whereas names representing
data are to be started with a lower case letter. The $.>$ symbolizes
the shell prompt. The first example of Fig.\ref{cat1}
shows the most basic use case: a simple service invocation.
Following the tradition of the UNIX shell, Drift uses the same
order of command and arguments, namely
$Service\ name_{1}\ ...\ name_{n}$.
Depending on the called service it would also be possible to
invoke services without any parameter names.

Since Drift is eager-evaluated the service shown in
Fig.\ref{cat1} is immediately started. In this case it's
a service called \textit{Cat} in reference to the widely-known
UNIX tool, which prints the content of given files to the screen.
So the \textit{Cat} service also returns the data of its arguments and
since there are only names in Drift, the \textit{Cat} services takes names
as input and returns the data associated with them.

Since the data returned by the invocation of the \textit{Cat} service is
not bound to any name, it just gets printed to the shell. When all
the data that is available behind the name(s) given to \textit{Cat} has been
printed, a new shell prompt is shown and further commands can be entered
by the user.

Fig.\ref{cat2} shows the same service invocation but this time the
result of the service call is bound to a name, using the assignment
syntax as known from other imperative languages. Again the
\textit{Cat} service is invoked immediately but since its result
is bound to a name, nothing is printed to the screen except a new
prompt so that new commands can be entered.

\begin{figure}[h]
    \centering
    \begin{subfigure}[b]{0.4\textwidth}

  \begin{lstlisting}
.> myresult = Cat mydata
.> $myresult
Lorem ipsum dolor sit amet,
consectetur adipiscing elit.
.>
  \end{lstlisting}

        \caption{Using the \$-operator to query the data behind a name.}
        \label{cat3}
    \end{subfigure}
    \hspace{20pt} %add desired spacing between images, e. g. ~, \quad, \qquad, \hfill etc.
      %(or a blank line to force the subfigure onto a new line)
    \begin{subfigure}[b]{0.4\textwidth}

  \begin{lstlisting}
.> Cat mydata
Lorem ipsum dolor sit amet,
[cancel]
.>
  \end{lstlisting}

        \caption{Canceling an ongoing data query.\\}
        \label{cat4}
    \end{subfigure}
    \caption{Basic query and query cancelation mechanism in Drift.}\label{drift-examples2}
\end{figure}

In order to query the data that is represented by a name, Drift offers
the \$-operator, a designated query operator, as shown in \ref{cat3}.
This query operator will receive \textit{all} the data available
behind a name and print it straight to the screen.

However, since services are invoked immediately and therefor executed
immediately, it is possible that the service has not yet produced
a single data item or has not yet finished processing and has
therefor not finished producing all of its output. This means that
it is absolutely possible that either no data or only some data is
shown, depending on what's available when the user issues the query.
\newline
Drift is not an \textit{all or nothing} batch system where the user
either sees no output, or all the output atomically. It's a streaming
system in which data is observable whenever it is available. This is
based on the observation that batch processing, even distributed
batch processing, is only \textit{a special case} of stream processing
and not the other way around, as was also proclaimed in
\cite[chapter 1]{flink}. This fits perfectly with the overall
approach of immediate feedback because not only the commands of
the shell are eagerly evaluated and because of that the services
immediately spawned, but also the data produced by the individual
components flowing through the system can be immediately observed
by the user.
\newline
Of course it would defeat the purpose and reactiveness of the system
if the user would have to wait for all the data to be finished, once
querying a name. Therefor query cancelation has been implemented as
shown in Fig.\ref{cat4}. When data is printed to the screen, either
because the result of the producing service invocation was not bound
to a name, or because a name was queried using the \$-operator, the
display of data can be canceled simply by pressing $q$ and new prompt
will be shown so that new actions can be taken.

Therefor printing data to the screen has not the role of presenting
real results, like in many batch processing systems. It rather becomes
a tool for the user to sneak a peek at what the system is currently
doing in order to assess whether or not the system is working
towards the desired result.
\newline

So where do all the available services and names come from?
In the same way that the UNIX shell does not provide much
built-in functionality except for only a very few built-in
commands, the Drift shell also does not provide any services.
Services need to be inserted into a \textit{service registry}.

This is a designated service invisible to the user that the
Drift shell assumes to be available at all times. When a command
is entered, the shell will consult the service registry about
whether the service exists and whether its usage corresponds to
its specification stored in the service registry.

However the shell does not \textit{download} the required service
in any way. It only fact-checks the service before issuing the
entered command as a task to the back-end system, as will be
described in chapter \ref{driftimplementation}.
If a service is unknown to the service registry, a
\textit{service unknown} error will be printed to the screen
and a new prompt is shown.
\newline

The availability of names is different. One of the design goals
of the Drift shell is to present the user with its own cloud so
to speak. Therefor any Drift shell invocation starts a new
\textit{session} using a UUID as a unique session identifier.
In the current implementation, this UUID is also generated and
handed out by the service registry.

Following this principle, any session starts with a
completely empty \textit{namespace} containing no names. However,
before the actual session starts and the entering of commands is allowed,
the user is presented with a setup phase. In this \textit{import phase}
she can only use the \textit{import} keyword to \textit{upload}
files from the local file system of where the Drift shell is run
into the Drift system.

When the user has finished the import phase, the actual session
begins and normal command input is possible. The \textit{import}
keyword and therefor any further imports are disabled.

\begin{wrapfigure}{l}{0.35\textwidth}
  \begin{lstlisting}
Session: e3d78ad5-898...
.> import data.csv
.> import test.txt
.> q
-------------------
.> ls
  data.csv
  test.txt
.> result = Cat test.txt
.> ls
  data.csv
  result
  test.txt
.> export result as result
.> q

  \end{lstlisting}
  \caption{Example of a short Drift session.}
  \label{drift-import}
\end{wrapfigure}

Fig. \ref{drift-import} shows an example session including the import
phase. Here two local files, namely \texttt{data.csv} and
\texttt{test.txt} are uploaded to the Drift back-end. When the
import phase is finished by pressing $q$, the normal session
begins. This example also introduces one of the few built-in
commands of the Drift shell, namely \texttt{ls}. These built-ins
are again lent from the original UNIX shell and UNIX file system,
so \texttt{ls} is of course used to list all the available names
in the current namespace.

Now with these imported names and associated data, services can
be invoked and their result can be bound to new names. In order
to recognize whether a name resulted from an import or was later
created during the session, only the imported names are allowed
to contain dots and file endings like \texttt{.txt}, \texttt{.csv}
or \texttt{.tar}.

In the same way that the import keyword during the import phase
allows to fill the initial namespace with names, the \textit{export}
keyword allows to \textit{download} the data referenced by a name
to the local file system under a given name. In the current
implementation this can be done at any time during a session
but it could also be a valid alternative to put the export
behavior into a seperate export phase before closing the session.

The local file in which the data is stored is placed inside a
directory with the same name as the session ID, eliminating the
possibility of name clashing between different sessions (of
possibly different users). However, the user of a session is
responsible for preventing name clashes from within a session
when using the \textit{export} keyword.
\newline

So far \textit{names} and \textit{services} and their basic
usage have been introduced. Initial data can be uploaded from
the local file system into the Drift back-end and imported as
names into the initial name space of a session of the Drift shell.
Resulting data from service invocations can be either printed
to the screen or bound to names using the assignment syntax known
from other languages and the namespace can be explored by using
well known UNIX file system commands like \texttt{ls}.

However, there are services whose result is not just singular data.
As has already been mentioned, one possible file extension for
imports is \texttt{.tar} and so naturally one service which is
able to deal with such imports is called \texttt{Untar*},
with the \texttt{*} on the right hand side being of significance
because it indicates that the output of \texttt{Untar*} is not
just data (a single file) but rather a bunch of names.
Which is why the concept of \textit{names} in the Drift language
is expanded to also include \textit{namespaces}, which directly
correspond to directories in the original UNIX file system.

Since the result of an invocation of \texttt{Untar*} is a
namespace, it wouldn't really make sense to be able to print
it to the screen. But it would also not make sense to bind
a bunch of names to only a single name, as was shown so far.

Therefor, the result of services returning a namespace must
also always be bind to a namespace.

\begin{wrapfigure}{l}{0.4\textwidth}
  \begin{lstlisting}
Session: e3d78ad5-898...
.> import comments.tar
.> q
-------------------
.> ls
  comments.tar
.> res/ = Untar* comments.tar
.> ls
  comments.tar
  res/
.> cd res/
.> ls
  c1
  c2
  \end{lstlisting}
  \caption{Example of a name space binding.}
  \label{untar}
\end{wrapfigure}

As shown in Fig.\ref{untar} \textit{namespaces} also start with
a lower case letter, just like names but namespaces must always
be trailed by a \texttt{/}. This and the additional \texttt{*}
allow the interpreter to syntactically verify that the issued
command is at least valid in terms of its basic format.
If the format is invalid an error is printed to the screen to
inform the user and the command is not accepted.

The resulting namespace then of course contains the unpacked
files as new names. This allows the namespace of the whole session
to become a namespace tree in the same way that the UNIX file system
offers a naming tree, with \texttt{/} as a delimiter. Using another
shell built-in, \texttt{cd}, the user can also \textit{walk} the
namespace tree just as with the usual UNIX file system, as is
also demonstrated in the example shown in Fig.\ref{untar}.

This is important because since there are no variables in the
Drift language, only names and namespaces one could now say
that either the original file system was liftet up into the
language or the language allows its variable-space to be
traversable like a tree. Both interpretations, I think, are valid.

But not only the variable-space or data-space forms a tree. As
was already mentioned, also the orchestration of services and
the data flow between them forms a tree or possibly even a DAG.

In order to allow for such composition, the Drift language also
features the \texttt{|}-operator (pipe operator) which behaves
exactly like the pipe operator already known from the UNIX shell
as introduced in chapter \ref{bash}. Fig.\ref{pipe-example}
shows one possible use of the pipe operator in order to
count the number of lines of comments that might have been
left under an online video or blog post.

This again showcases the \texttt{*} syntax, which is not only
needed in order to indicate that a service \textit{produces}
a namespace but also to indicate that a service \textit{receives}
 a namespace as input. Therefor the interpreter can immediately check,
(with help of the service registry)
whether or not the basic formats of names, namespaces and services
are valid. But this is not only supposed to help the interpreter.
It is of course done with the best intentions for the user, so that he,
given meaningful names and these basic format identifiers, might
intuitively understand even longer chains.

\begin{figure}[h]
    \begin{lstlisting}
Session: e3d78ad5-898...
.> import comments.tar
.> q
-------------------
.> linesOfComments = Untar* comments.tar | *Concat | LineCount
.> $linesOfComments
  4653
.>
    \end{lstlisting}
  \caption{Example of using the pipe operator and \texttt{*} syntax.}
  \label{pipe-example}
\end{figure}

However, it is not
immediately obvious why such an operator is actually needed.
As was shown so far, the result of every single service invocation,
whether it returns singular data or a namespace, could be bound to
a name or namespace respectively. Leaving it at that, would create
the same zick-zack pattern as demonstrated in Fig.\ref{cf-example}
in chapter \ref{cuneiform}. Unfortunately this would lead the user
into a direction that is opposite to the core philosophy of the
Drift language.

In Drift, there are only names. Not because this makes things
easy but rather because it is the \textit{names} that carry all
the semantics for us programmers and the goal of the Drift language
is to eliminate every unnecessary clutter and focus on the essence,
focus on the names and chose them wisely in order for them to
convey as much \textit{meaning} as possible.

But not every single service invocation is full of meaning. Sometimes
a certain combination of service calls, maybe even in the specified
order provide a meaningful transformation. Instead forcing the user
to provide meaningful names for every single invocation would lead
to meaningless placeholder names like \texttt{ccd} or \texttt{idx2}.

So in the same way, that \textit{Git} puts the responsibility of
chosing when and how much to commit into the hands of the user,
the \texttt{|}-operator allows the user the chain together
multiple service invocations sequentially and then giving the
result of the whole chain a meaningful name. So it is still
possible to revert to single-stepping each service call if
deemed necessary but arbitrary batch sizes of services are
also possible. It is up to the user to chose wisely.

\begin{wrapfigure}{l}{0.4\textwidth}
  \begin{lstlisting}
Session: e3d78ad5-898...
.> import data.csv
.> q
-------------------
.> a = A data.csv
.> b = B a
.> c = C a
.> d = D b c
  \end{lstlisting}
  \caption{Example showing how to construct dependency graphs.}
  \label{dag}
\end{wrapfigure}

But sequential chains of service do not make for full fledged
DAGs. Fig.\ref{dag} shows an example of the well known
diamond formation. So simply reusing the names that have
been created through earlier service calls and binding their
results to new names in order to make them reusable later allows
for the creation of arbitrary dependency graphs.
These graphs naturally lend themselves to be visualized as will
be shown in chapter \ref{driftui}.

So far, one could argue that besides the eager evaluation,
the Drift language behaves like a basic functional language.
There are no variables so there is no contention about the values
of such variables (the name \textit{variable} itself suggests
change over time, also known as state) and data produced by services
is only loosely bound to names as is mostly done in functional languages
\footnote{by the use of a so called \textit{let}-expression \cite{let}}.

\begin{wrapfigure}{r}{0.4\textwidth}
  \begin{lstlisting}
.> a = A data.csv
.> b = B a
.> a = C
.> ?b
  B a
.> ?b.a
  A data.csv
.> ??
  B (A data.csv)
.>
  \end{lstlisting}
  \caption{Example showing state in the language and how service
           invocations behave as closures.}
  \label{state}
\end{wrapfigure}

However, Fig.\ref{state} shows how state is included in the Drift
language because names, even when services have already been started
with these names as input, depending on the data they represent,
can be overwritten and bound to new data any time.

The rather abstract example shows how a name \texttt{a} is created
by starting the service \texttt{A} with the input name \texttt{data.csv}
which can savely be assumed to be an imported name, because only import
names are allowed dots and file endings like \texttt{.csv}.
While this \texttt{A} service might be running, another service,
\texttt{B} is started, using the ouput of \texttt{A} as input.
Since data is made available to these names as soon as it is available
there might be a data stream between these two services, who also
form a sequential data dependency. But it is also possible, that the
user enteres line 2 after the invocation of service \texttt{A} in line
1 has already finished. Then all the data is available and \texttt{B}
can pull that data depending only on its own processing speed.

Then, in line 3, the name \texttt{a} is re-bound to the result
produced by the invocation of \texttt{C}. So when now querying
the data behind \texttt{a} using the \texttt{\$}-operator, the
user would see the data produced and streamed by \texttt{C}.
If the user queries the data available behind the name \texttt{b}
however, it would see the result of B \textit{still consuming
from \texttt{a}}. Therefor the invocation of \texttt{B} on \texttt{a}
\textit{captures} the value of \texttt{a} indefinitely and is therefor
not concerned by any later change. Therefor service invocations
behave exactly like \textit{closures}, a concept from the area of
functional programming describing function invocations that capture
their environment (variable names and their values) \cite{closure},
\cite{closurewiki}.
\newline

Unfortunately this can make it difficult to remember how the data
currently shown by a query on a name came to be.
In order to help with that, the Drift Shell offers another built-in
feature, namely the \textit{history query} operators \texttt{?} and
\texttt{??}.

As shown in line 4-7 in Fig.\ref{state}, the simple history query
operator \texttt{?} can be used for a quick resolution of the
statement that was issued in order to produce the data that is
currently bound to a name. In the case of the name \texttt{b}
that statement was the invocation of \texttt{B} on \texttt{a}.

But since the value of \texttt{a} has changed since the original
invocation, the simple history query operator \texttt{?} can further
be used to resolve \textit{any} name in a statement and show the
statement that created that name, at the point in time when the
overall statement was issued, as shown in line 6 and 7. When
the overall statement was issued, the name \texttt{a} was bound
to the result of the invocation of \texttt{A} on \texttt{data.csv}
and not to the result of \texttt{C} as is now the case.

So the simple history query operator \texttt{?} can be used to
arbitrarily traverse the whole history tree of any given name
or statement. The full history query operator \texttt{??} is then
a mere shortcut that immediately unveils the complete history tree
of a statement. This should be used carefully, since
all the intermediate names are being resolved, it is up to the
user to reconstruct the meaning of the issued statements, as shown
in line 9.
\newline

\begin{wrapfigure}{l}{0.4\textwidth}
  \begin{lstlisting}
.> a = A data.csv
.> b = B a
.> ls
.> rm a
.> ls
  b
  data.csv
.> ?b
  B a
.> ?b.a
  A data.csv
.>
  \end{lstlisting}
  \caption{Example showing how names can be removed.}
  \label{remove}
\end{wrapfigure}

One interesting aspect that follows from this is shown in Fig.\ref{remove}.
Since service invocations capture the values of their input names
so that they can be overwritten, these names cannot only
be overwritten but also flat out deleted. In order to that, the
Drift shell offers yet another built-in command called \texttt{rm}.

Although the deleted name is no longer shown by \texttt{ls} and
therefor currently not available for any commands, its value is
still captured by the invocation of \texttt{B} just as it was
captured in Fig.\ref{state}.

In that sense the Drift language allows for state, because the value
of its variables, its names, depends on the moment of time when they
are queried. But it also works like \textit{Git} in the sense that
the user can observe her behavior immediately: things that are deleted
are gone but underneath everything depending on those things still
works.
And so the next chapter will show how these semantics allow for
and are used for building a distributed system, orchestrating
multiple independent services on multiple machines in a fault-tolerant
manner.


