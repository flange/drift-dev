\section{Distributed Programming}
\label{distributedprogramming}

In 1937 a paper titled \textit{``On computable numbers, with an
application to the Entscheidungsproblem''}, written by Alan Turing, was
published \cite{turingcomputable}. In it Turing defines what is now
known as a \textit{Turing machine}, a theoretical machine model for
executing arbitrary computation. He demonstrates how to formulate
instructions for this machine in order to configure its execution
behavior and goes so far as to show how this
machine could even receive another machine configuration as its input
and then execute what the other machine would have done, thereby
effectively emulating the given machine.

But he wasn't the only one thinking about computation. As summarized
by Denning in \cite{whatiscomputation}, Kurt Gödel, Alonzo Church and Emil
Post all contributed to the task of exploring the boundaries of
computation, setting the tone for the next 80 years of research
concerning \textit{computation}, \textit{computers} and \textit{automation}.
\newline


Of all the proposed approaches of how to tackle the concept of computation
and turning it into a tool that could be understood and used by delivering
a theoretical framework, one could argue that Turing's machine model
emerged as a winner because, although not at the time, it turned out
to be closest to what machines became to be. How they were structed and
how they basically operated. Still, to this day.

So in order to use or \textit{program} these machines, these computers,
programming languages were developed and over the decades have gone
through their own evolutionary process \cite{pl-gens} with the
first generation using binary machine instructions to todays languages
that allow for higher level programming styles like
\textit{object oriented programming } \cite{bjarneOO} or
\textit{functional programming} \cite{wadler-functional}.













































