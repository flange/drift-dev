\subsection{Distributed Objects}
\label{DistributedObjects}


So how did the tools, the programming languages we use, react
to this paradigm shift? Well, Fig.\ref{rmi-example} shows an exemplary
client and server implementation using a framework called
\textit{Java Remote Method Invocation} (RMI) which implements the
more abstract request-and-response concept of
\textit{Remote Procedure Calls} (RPC).

\begin{figure}[h!]
    \vspace{5mm}
    \begin{subfigure}[b]{0.99\textwidth}

    \begin{lstlisting}
public class ServerOperation extends UnicastRemoteObject implements RMIInterface {

    private static final long serialVersionUID = 1L;

    @Override
    public String helloTo(String name) throws RemoteException {
        return "hello to " + name;
    }
}
    \end{lstlisting}

      \caption{Java RMI server providing the remote method 'helloTo()'}
      \label{fig:rmi-server}
    \end{subfigure}
    \hfill
    \vspace{10mm}
    \begin{subfigure}[b]{0.99\textwidth}

    \begin{lstlisting}
public class ClientOperation {

  private static RMIInterface look_up;

  public static void main(String[] args) {

    look_up = (RMIInterface) Naming.lookup("//localhost/MyServer");
    String response = look_up.helloTo("Frank");
  }
}
    \end{lstlisting}

      \caption{Java RMI client calling the remote method 'helloTo()'}
      \label{fig:rmi-client}
    \end{subfigure}

  \caption{Java Remote Method Invocation example}
  \label{rmi-example}

\end{figure}


The underlying concept that tries to unify object oriented programming
and a distributed or networked system is often summarized as
\textit{distributed objects} \cite{distobjects}. As the names suggest,
in this programming model, objects cannot only exists in the local
address space but can also reside on remote machines with their
methods hence being invoked remotely which by itself is not a bad idea.

But when one looks at how languages and frameworks implemented
this new model one can see the same pattern as with multithreading
libraries because the remote functionality and remote method calls are
implemented by inheriting and overwriting certain methods on the server
side and on the client side by instantiating a remote object and calling
a remote method on that object. Unfortunately using the exact same
syntax as for any local function call.

The difference of course is, that the remote procedure call, being remote,
has to deal with different error and failure scenarios than any local
method call. Hence it might throw some sort of remote exception which,
from the view point of the client can be totally suprising, because
syntactically there is no indication whatsoever that this is an action
that involves the network or any remote machine.

The lesson here of course is the same as with concurrency support in
programming languages because the underlying system changed in nontrivial
ways but the distributed objects community tried to apply the same
object oriented programming concepts and languages to this new domain,
neglecting any additional and unique problems of this new setting and
hiding all new functionality behind core language features, so that
any new semantics were only available on a napkin and tool support
was nearly impossible.
\newline

Unfortunately today, one could argue that the same approach is being
used again in the context of highly distributed systems built after
the microservice paradigm \cite{microservices}. Fig.\ref{pys3} shows
an example of a Python binding for using the
\textit{Amazon Web Services} (AWS)
service called \textit{Simple Storage Service} (S3) \cite{aws}, \cite{aws-s3}.

\begin{figure}[h]
    \begin{lstlisting}
import boto

s3 = boto.connect_s3()
bucket = s3.create_bucket('media.yourdomain.com')

key = bucket.new_key('examples/first_file.csv')
key.set_contents_from_filename('/home/frank/first_file.csv')

key.set_acl('public-read')
    \end{lstlisting}
  \caption{Amazon AWS S3 storage example using the Python interfacing
          library 'boto'.}
  \label{pys3}
\end{figure}

This service is supposed to provide a remote storage solution,
by storing arbitrary data objects behind unique keys. As can be seen
in the official example \cite{pys3-example} depicted in Fig.\ref{pys3}
the unique names are being written using the same format as the widely
sed UNIX file systems. The example shows how a local file is being
uploaded to the service. This showcases that Amazon S3 is mostly used
as a service providing a remote file system to its clients.

But again, from the language perspective of the Python interpreter,
these remote calls are just normal \textit{local} function calls
because the semantics and functionality of the S3 service only
exist in the informal documentation provided by Amazon without any formal
specification and without any chance of automated tool support
regarding syntax (parameters) or semantics.

This might be acceptable from a business perspective because already
existing systems can easily be extended to use these new services
but this puts all the responsibilities in the hands of the programmer
and also enables vendor lock-in because the semantics are defined
by the company providing the abstractions and not only can they
change them whenever they choose, but also other companies can have
a vastly different APIs and semantics.
