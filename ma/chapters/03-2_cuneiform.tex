\subsection{The Cuneiform Language}
\label{cuneiform}

One area that features a wide variety of custom tailored languages
which allow to describe a possibly distributed system from a
bird's-eye perspective, is the field of \textit{scientific workflows}
\cite{kepler}, \cite{swf-survey}.
Especially in the context of \textit{bioinformatics}, scientific
workflow systems like Kepler \cite{kepler}, Taverna \cite{taverna}
and Pegasus \cite{pegasus2004}, \cite{pegasus2005} have become an
invaluable tool because they allow to utilize distributed systems
in order to cope with the sheer amount of data that's been used and
might also allow for massive gains from parallelization.

One language that is beeing developed here at the Humboldt
Universität zu Berlin, at the department of \textit{Knowledge Management
in Bioinformatics}, is called \textit{Cuneiform}
\cite{cuneiform}, \cite{saasfee}.

Matching the approach advocated by Rich Hickey and as presented in
the chapter \ref{LanguageOfTheSystem}, Cuneiform is a functional
workflow language with a formal foundation in the lambda calculus
\cite{lambdachurch}, \cite{lambda-calc} and with an emphasis on
immutable data. Another key aspect of Cuneiform is its
\textit{foreign function interface} which allows to use already
exisiting bioinformatics tools with any modifications. Most
other scientific workflow systems force users to reimplemented
their desired services using the abstractions given by the framework.

\begin{figure}[h]
    \begin{lstlisting}
deftask untar (<list(File)> : tar(File)) in bash *{
  tar xf $tar
  list=`tar tf $tar`
}*

txt = untar(tar: 'corpus.tar');
csv = wc(txt: txt);
result = groupby(csv: csv);
result;

    \end{lstlisting}
  \caption{A shortened word count example using the functional workflow
           language \textit{Cuneiform}}
  \label{cf-example}
\end{figure}

Fig.\ref{cf-example} shows a shortened and small word count example,
taken directly from Brandt et al. \cite{cuneiform}. First this
example shows how a Cuneiform task is defined using the \textit{deftask}
keyword. The task named \textit{untar} is supposed to deliver the
same functionality as the widely known UNIX tool \textit{tar} and
therefor expects a \textit{file} as
input and returns a \textit{list of files} as output. Showcasing the
foreign function interface capabilities, one can see that it is
specified that the body of this task will actually be written in
\textit{Bash}, the language of the same-named UNIX-like operating
system shell.

The actual workflow description is shown in lines 6-9. Since
Cuneiform uses \textit{lazy evaluation} the first lines 6-8
don't actually trigger any behavior of the system. They only
describe the workflow. Line 9 shows the query syntax of the
system, which will finally trigger the actual execution of the specified
workflow and print its result.

As one can see the results of the called tasks are bound to names
using the \textit{=} operator syntax, which in an imperative language
might be called the \textit{assignment operator}. Since Cuneiform follows
the functional programming paradigm, these name bindings, rather than
being assignments, are later dissolved using textual substitution.

Each name in the example work flow is used by the following task
invocation as an input, therefor creating a zick-zack like pattern
which basically describes a sequential data dependency between these tasks.

This shows that the coordination expressed via Cuneiform describes the
data dependenies and therefor the \textit{data flow} between tasks which
cannot only be a sequential flow, but rather can branch out and join
again in any possible way, describing a \textit{directed acyclic graph}
(DAG).

Since Cuneiform uses lazy evaluation, lines 6--8 build the DAG which
can then be optimized or minimalized when the actual query operator
in line 9 triggers its execution.
\newline

The next chapter will try to show, how these core concepts of a
functional language and its data flow describing qualities can be
mapped or emulated by the language used in the common UNIX shell.
