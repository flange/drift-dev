\section{Drift}
\label{driftintro}

The last chapter, chapter \ref{RelatedWork}, introduced multiple
seemingly unrelated software projects and also the people behind
them and their ideas and principles. However unrelated these projects seem,
they each contribute to the work presented in the following
chapters, namely the \textit{Drift} project.

The first idea that was introduced in chapter \ref{LanguageOfTheSystem}
was the need for a language of the system. A language that would allow
to orchestrate a possibly distributed or even microservice based system
from a bird's-eye perspective. In order to allow such a system's
perspective without depending on global state in the distributed
context and to keep the complexity arising from the combinatorial
explosion of contending and interleaving interactions of the
autonomous and independent components of the system to a minimum,
the importance of immutable data in the form of value-oriented
programming was introduced.

Chapter \ref{cuneiform} showed a modern contender for such a
language and system, namely Cuneiform: a functional scientific
workflow language, utilizing most of the functional language
toolbox like lazy-evaluation, single-assign name bindings and
independent subexpression evaluation as defined by the
Church-Rosser Theorem \cite{churchrosser}, \cite{churchrosserwiki}.
However, using Cuneiform and similar systems means fully accepting
the paradigm of functional programming and abandoning any
imperative roots of programming, which seems rather radical given
the wide adoption of imperative languages which shaped the
way people think about programming for generations.

In order to look for alternatives, chapter \ref{bash} went back
in time and took a closer look at a system that has already been
around for decades, namely the UNIX operating system. The interesting
aspect when it comes to system orchestration and coordination from
a bird's-eye perspective is of course the UNIX shell. Given
that the operating system allows for multiple independent
processes to run seemingly autonomous, the shell not only provides
mechanisms to spawn these processes but also to connect
the data flow between them using well-known abstractions like pipes
and files. Furthermore, the chapter \ref{bash} showed, that when taken
seriously, the constructs offered by the shell language can even
be mapped to functional language concepts as used by languages
like Cuneiform. This revealed that the UNIX shell offers roughly
the same capabilities in terms of orchestrating a system of
independent processes \textit{plus} being imperative, widely-known
and interactive.

Especially the interactiveness delivering immediate feedback is
something that provides multiple benefits as was introduced in chapter
\ref{bretvictor} and was something that for the longest time was second
nature in the realm of functional but not imperative programming.

The ``big problem'' so to speak, when it comes to the design of the
UNIX shell as well as the operating system itself and the idea of
porting it to the context of distributed systems is of course the
file system or more generally the data layer. Since its design
follows the place-oriented paradigm as introduced in chapter
\ref{LanguageOfTheSystem}, any subsequent implementations
using distributed file systems have stuck to the same paradigm
and therefor run into the same issues.

So in the same way that chapters \ref{cuneiform} and \ref{bash}
are supposed to challenge the widely accepted believe of
incompatiblity between functional and imperative language concepts,
chapters \ref{datomic} and \ref{git} try to challenge the
preconceptions of distributed immutable data systems and their
implications. So chapter \ref{datomic} introduces Datomic,
a distributed immutable facts log when viewed from the user
perspective, but when looked at from the implementation perspective,
Datomic only creates the illusion of immutability by putting a thin
immutability layer on top of widely-known and already existing
mutable distributed storage solutions. Git on the other hand,
as introduced in chapter \ref{git}, is the complete opposite. It
provides the user with a mutable and stateful perspective on her
own file system built on top of an immutable content addressable
file system underneath.

This again challenges the widespread believe that immutability
or statefulness are system design primitives that define the
system to its core and must be applied throughout the whole system
stack, up to the user interface.
\newline

So naturally the challenge arose, of whether or not it would be
possible to build an imperative and distributed
shell language with a flat learning curve using well known
imperative concepts  but with the same orchestrational powers and
interactiveness as provided by functional languages on top of
an immutable distributed file system in order to utilize the
benefits of the value-oriented approach and to keep the contention
and complexity of the whole system to a minimum.
\newline

The following chapters will present the current result of that challenge
by first introducing the \textit{Drift language}, a tiny and abstract
microservice coordination language built on the concepts of names,
namespaces and services. Then its distributed implementation including
the immutable \textit{Drift file system} will be presented, which
provides the illusion of a file system by utilizing distributed
message queues. Finally the \textit{Drift UI} will be presented
which is implemented as a web interface for any modern browser
and contains not only the shell but also a visual live-representation
of the system state using the well known Petri Net syntax as well as
a time bar that allows to revisit old system states.


