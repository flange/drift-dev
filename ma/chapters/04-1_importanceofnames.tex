\subsection{The Importance of Names}
\label{importanceofnames}

Developing a programming language can be an overwhelmingly
ambitious task. Most often it's only that when one sets out
to do so, that one realizes how many features even
basic languages offer: arithmetic, different types of loops,
conditionals, logical operators, built-in data types and a
type system that allows for arbitrary user-defined types.
Depending on the overall paradigm the list might continue
with things like classes, objects, inheritance, polymorphism
and generic types or things like type classes, higher-order
functions, partial application or monads and do-notation.

Although it would have been nice to have at least scratched on
some of these aspects, it became clear right at the beginning
that most of these features would be left untouched given the
scope and resources of the work presented here.

But that is not necessarily a bad thing. Constraints often times
have the benefit that they force resources to be spent on what
really matters and so in that same spirit the goal for the
Drift coordination language was to carve out what is
\textit{essential} to programming. It could almost be said that
the goal was to find the smallest possible language that
would not allow for any construct to be taken out and still be
offering the same possibilities.
\newline

So what is the essence of programming?

Well that depends on what the user is supposed to be able to express
in that language. If it's computation through calculation, built-in
arithmetic operators like $+$ or $*$ might seem a good idea. If
the user should be able to express things that model the real world,
like in many simulation contexts for example, having constructs
that represent real world objects and allow for equipping them with
properties like color or weight could be another sensible approach,
built on basic arithmetic. This would naturally lead to built-in
data types for which these basic arithmetic operators are
defined and would build up to a type system which could deal with
arbitrary user-defined types and the operations defined on them.

But defining these structures, functions or classes and modelling their
properties and functionalities is only one part of constructing
a program because these structures in and by themselves have nothing
on which they could execute their behavior on and are not connected
in any way in order to produce a final output as the result of
multiple functionalities chained together.

A program does not only exist as the mere sum of its parts but
also of a description of how these parts interact and compose and
where data comes from and where it goes which is what is mostly
described inside the \textit{main-method} which virtually
all programming languages independent of their paradigm have in
common because it serves as the universal starting point of a program.

Most languages use the same syntax and semantics
for both purposes: for describing the individual components and for
describing their often times more abstract orchestration.
This is demonstrated by the examples shown in Fig.\ref{c-and-java}.

\begin{figure}[h!]
    \vspace{5mm}
    \begin{subfigure}[b]{0.41\textwidth}

    \begin{lstlisting}
#include <stdio.h>

struct point {
  int x;
  int y;
};

int point_sum(struct point p) {
  return (p.x + p.y);
}

int main() {

  struct point p;
  p.x = 3;
  p.y = 5;

  printf("%d\n", point_sum(p));

  return 0;
}
    \end{lstlisting}

      \caption{Component definition and orchestration in C.}
      \label{fig:c-example}
    \end{subfigure}
    \hfill
    \vspace{10pt}
    \begin{subfigure}[b]{0.54\textwidth}

    \begin{lstlisting}
public class Point {
  private int x;
  private int y;

  public Point(int x, int y) {
    this.x = x;
    this.y = y;
  }

  public int getSum() {
    return (this.x + this.y);
  }

  public static void main(String[] args){

    Point p = new Point(3, 5);
    System.out.println(p.getSum());

    return;
  }
}
    \end{lstlisting}

      \caption{Component definition and orchestration in Java.}
      \label{fig:java-example}
    \end{subfigure}

  \caption{Examples of component definition and orchestration done in
           languages that use the same language for both tasks.}
  \label{c-and-java}

\end{figure}

The left hand side figure, Fig.\ref{c-and-java}a, shows an example
using the C programming language. In it, a new compound user-defined
data type \textit{point} is declared and defined, consisting of two
integers representing the x and y coordinates of a point.
Then a function is declared and defined which takes such a point
as an input and returns the sum of its coordinates.

These are the blue prints of the components of this tiny system.
Inside the main-methode an \textit{instance} of a point is created and
its coordinates are set to the values 3 and 5, which are part of
the source code itself. Then a function call is issued using a copy of
this point instance which creates a single instance of the mentionend
function whose return value is then immediately printed to the screen.

So inside the main method the actual \textit{business logic} of
the program is described by composing and chaining instances of the
described components and inserting data into the system and letting
it flow through the components in order to produce the desired output.
\newline

The same thing happens in the example on the right hand side, in
Fig.\ref{c-and-java}b, this time using the object oriented
programming language Java. In this example a class \textit{Point} is
defined, also using two integers to represent its coordinates.
The class defines two methods: a constructor method for ease of
instantiation and a \textit{getSum(int, int)} method to again
calculate the sum of the coordinates.

As with the C example on the left, this plain class definition
by itself would not make for much of a program. The business logic
is again defined in the main-method, where an instance of the Point
class, a Point object, is instantiated (again with data contained
inside the source code file itself) and the result of a call to
its instance method \textit{getSum} is printed.
\newline

However, it is possible to disentangle the component description
from the higher level component composition. One example of this
has already been introduced, namely the UNIX stack as shown in
Fig.\ref{unixstack}. In this stack, each component is a UNIX
process, dealing only with the task of processing the data its
been given and producing its output respectively. The higher-level
business logic of the system is then defined using the shell
which can be used to instantiate components (processes),
just like objects are instantiated in Java, and describe the data
flow between them. The difference is, that the shell language, the
\textit{coordination language} is completely
different compared to the \textit{computation language} used to
describe the computation done by each individual process.

So which of the two, computation or coordination, is actually
more relevant in the context of programming a distributed system?
Well, how would one distribute a computation like $1 + 2$?
Computation in and by itself has something local about it because
the operator needs to have perceived or received its operands in
order to carry out its computation. The \textit{distributedness}
actually emerges because although each individual operation is
carried out on a local machine, the actuall composition of multiple
operations and data flow through them spans multiple locations, i.e.
multiple machines.

So as I have already tried to argue in chapter \ref{LanguageEvolution},
for the context of distributed systems, it's not so much computation
that defines the programming of distributed systems (because
of computation having this inherently local aspect to it, we
can for the most part reuse the already existing computation languages)
but rather \textit{coordination} and \textit{communication} between
the components and operations residing on different machines.
\newline

So the question of what is the essence of programming
changes in the context of distributed systems and therefor for the
context of this work to: what is the essence of composition?
\newline

To me, as far as I can work out, the essence of composition
is simply \textit{data} and \textit{functionality}. Whether
functionality is provided by plain functions modelled after
the example of mathematical functions, or with objects that
encapsulate functionality via methods, or with actors or
processes or microservices is not really important.

I believe the user needs these two things: she needs to be
able to identify and talk about her data and she needs to be
able to identify and invoke her services with that data.
This also includes identifying the data that might be the
result of such a service invocation.

So what's the most natural way for \textit{people} to identify things?
By giving them \textit{names}. Names that carry \textit{meaning}
and therefor provide semantics (for humans).
\newline

\textit{Meaning} is an interesting word and one can easily find examples
where it is ascribed to mechanisms that do not really provide it.
For example one of the fundamental books on programming language
semantics and type systems, \textit{Types and Programming Languages}
by Pierce \cite{pierce} contains the following quote by Mark Manasse
on page 208:

\begin{quote}
The fundamental problem addressed by a type theory is to ensure
that programs have meaning. The fundamental problem caused by a
type theory is that meaningful programs may not have meanings ascribed
to them. The quest for richer type systems results from this tension.
\end{quote}

In this quote, type systems are identified as the source of
meaning for programs and there is some truth to that. But I
would like to argue that it is not the type system itself that
provides the meaning, but rather the \textit{names} of the types
that do so.
\newpage

\begin{wrapfigure}{l}{0.54\textwidth}
  \begin{lstlisting}
void runServer(ServerSocket serverSock) {
  serverSock.listenOnPort(9000);
}
  \end{lstlisting}


  \begin{lstlisting}
void runServer(Uhjgj abcdeg) {
  abcdeg.inufdhg(9000);
}
  \end{lstlisting}

  \begin{lstlisting}
void runServer(143 abcdeg) {
  abcdeg.inufdhg(9000);
}
  \end{lstlisting}
  \caption{Example showing the importance of \textit{names} for
           human understanding and reasoning about code.}
  \label{type-names}
\end{wrapfigure}

Consider for example the three versions of the same function in
Fig.\ref{type-names}. The first version represents the kind of
code that is used in software development today, using meaningful
names not only for the data types but for the variable names as well.
This makes it easy for humans to understand why a server socket
has a method that triggers listening on a port and why this method
receives a number as input. Most of the meaning is conveyed through
the names, if chosen appropriately.

In the second version the names are obfuscated but still used as
expected by the compiler. This will happily compile, but most of its
meaning is gone. The third version reduces the type identifier to
a number. Most language grammars do not allow for type names to
be numbers, so this won't compile but the type checking engine itself
would probably still accept it. Because to the type checker, the type
is just an ID and as along as it can check whether or not the type
with the ID 143 provides a method with the used identifier and
signature, it would not raise any flags.
\newline

The \textit{meaning} is not conveyed by types. It's conveyed
by the \textit{names} of the types and of course the names of
variables, methods, classes, objects, and so on.
\newline

\begin{wrapfigure}{l}{0.4\textwidth}
  \begin{lstlisting}
int add_one(int n) {
  return (n - 1);
}
  \end{lstlisting}
  \caption{Example showing a bug that cannot be detected by
           the compiler or type checker.}
  \label{add-one}
\end{wrapfigure}

Consider another example, shown in Fig.\ref{add-one}. This example
shows a simple function whose name suggests that it returns its
argument incremented by one. Unfortunately its code does the exact
opposite and returns the decremented input. This will also
happily compile because \textit{add\_one} is a valid function identifier
and \textit{n - 1} is a valid operation on integers. But we as people
will immediate spot the mismatch between the meaning of its name and
its provided functionality.

So when one tries to reduce coordination down to its core parts,
namely data and services, the essential abstract thoughts that can
be expressed by the programmer are something like ``do this on that'',
or ``do this on that and put the result there'', or ``do this
on that and then that and put the final result over there''.
Interestingly even on such an abstract and rather comical scale,
these abstract statements can be categorized into two groups.
They are either describing a functionality chain, where each
component depends on the output of its predecessor and they
all form a sort of pipeline through which the data flows
sequentially or the services are completely independent of
one another.

So naturally the formulation of data and services forms a
dependency tree or, depending on the allowed complexity,
a directed acyclic graph (DAG). Which can be used effectively
for visualizations as will be shown in later chapters.
