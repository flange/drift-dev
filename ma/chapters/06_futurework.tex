\section{Future Work}
\label{futurework}

As was summarized by the last chapter, the
\textit{Drift Programming Environment} is a collection of different
components ranging from a coordination language and interpreter
to an abstract distributed file system and graphical user interface.

Since each of these components should be seen as a prototype
or proof-of-concept, they naturally leave much to be desired and
explored. So this chapter will discuss further developments of
each component which were deemed out of the scope of the work
presented here due to time and other resource limitations.
\newline

Starting with the Drift coordination language, one of the next
steps would be to provide a formal specification of the language.
So far the language has only been defined in terms of its implementation
and some prose, trying to explain its concepts. A formal semantics
would be needed in order to decouple the definition of the language
from its current implementation but would probably also allow
further reduction and understanding of the language constructs.
Moreover, the semantics of the language would naturally lend to
also providing a type system, which could further enhance the
meaning given appropriate names and allow for sound guarantees
in terms of stuckness or error handling.

Furthermore, in order to fully evaluate the helpfullness of the
language overall, user studies would be needed to assess whether
or not the rather abstract and concise syntax and semantics of
the language suffice in order for users to orchestrate their
distributed systems.

The Drift shell could be improved in terms of its error output
but also in terms of its input handling, providing features
like autocompletion for commands or a command history. Additionally,
as was already mentioned in chapter \ref{driftshell} it could make
sense to include a service info cache into the shell, which would
reduce the number of round trips to the service registry in order
to assess whether the service used in a given command are used
correctly.
\newline

The Drift back end could be improved immensely at almost any
aspect. One of the most obvious of these aspects is the inability
of the system to deal with its own resource limitations. As was
already hinted at in chapter \ref{LanguageOfTheSystem}, systems
build on the idea of immutable data tend to not scale in terms of
space. In this case the only space that accumulates over time
is memory used to store data (messages) in the message queues.
The system assumes that there will always be enough memory to
allow the user to create new names and therefore spawn new
messages while indefinitely keeping every other message queue
ever created.

This could be dealt with by garbage collecting, e.g. deleting,
queues where every active user session has either deleted or reassigned
the name that was once bound to that particular queue (reference counting).
This would eliminate the caching effect and would force future tasks
to recompute the content of the queue even though its result
was already computed once, but would on the other hand allow the system
to continue to function and not run into memory issues.

Another aspect of garbage collection concerns services that
are running even though their result is currently never used.
Imagine for example a user issuing a command, immediately querying
its output and seeing that she made a mistake. The user can then
immediate reassign the name to a new service invocation but the
old service invocation is still running, even though its result
will probably never be used. So canceling the tasks that are
executed unnecessarily would free up the workers to execute tasks
that are actually needed and would therefore increase the throughput
of the system.
\newline

Another issue concerning workers is scheduling, which of course
is a huge research topic in and by itself. The current implementation
uses round-robin scheduling by the task queue, distributing incoming
tasks evenly among all available workers. This was deemed
reasonable under the assumption of a homogenous cluster in terms
of hardware. But given slight variations in terms of available
CPU power, memory or bandwith, this could of course be
replaced by a scheduling component that takes all these different
aspects into account. Maybe even learns or adopts to different
workloads on the fly.

Additionally another aspect of the scheduling concerns the
order in which tasks are scheduled. The current implementation
treats all tasks as equal, whether or not they are considered
producers or consumers. So for the sake of the example one could
imagine a system with only two workers $W_{1}$, $W_{2}$ but
4 tasks queued in the order: $C_{2}$, $P_{2}$, $C_{1}$, $P_{1}$
with the right most task, $P_{1}$, being the next to be executed.
Given the two workers, $W_{1}$ would execute $P_{1}$ and $W_{2}$
would execute $C_{1}$, effectively idling, waiting for input
from $W_{1}$.

However, if the system would schedule all the producing
tasks first, $W_{1}$ would still execute $P_{1}$ but $W_{2}$
would now execute $P_{2}$, therefore really computing in parallel
and always guaranteeing that whenever a consumer task is executed
the input data for that consumer is always already available.
But this would not only mean reordering the task queue, which
might not be that difficult but also infering which task is
categorized as a producer and which as a consumer, which might
not at all be trivial or even possible to figure out.
\newline

Given the web interface I personally don't see that much room
for improvement. Of course, this being only a prototype probably every
button or element could be implemented nicer and prettier given all the
features the modern web has to offer. However, the only feature I would
actually really add besides much overdue cosmetic work, is for
the user to be able to write a certain system snapshot to a file
and maybe load them from a file in order to resume from that
snapshort, either for backup purposes or simply convenience
when moving between systems. But given the sandbox nature of
the browser and its JavaScript engine due to important security
concerns when browsing the web, this seems rather unlikely.
\newline

These conclude the major implementational weak points which
would likely need to be addressed in any further development
of the language and system.

Otherwise I can only encourage future projects to challenge
wide-spread believes about imperative or functional languages
or mutable and immutable data, especially in the
context of distributed systems, by trying to emulate one with
the other in order to gain further knowledge and understanding
about which concepts are needed on which layer of the system
stack to fully utilize their advantages and to balance out their drawbacks.


