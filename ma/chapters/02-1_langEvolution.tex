\subsection{Language Evolution}
\label{LanguageEvolution}

But no matter what mainstream programming language one looks at today,
they all seem to have one thing in common, which also gets replicated
by each new language that arrives: they are all focussing on
describing computation. Computation executed by a single core machine.
Because for most of the last century, this was the problem we were
facing.

However, with the advent of commodity multicore hardware
\cite{core2duo} the problem domain changed. Today even a single a machine
isn't a single machine anymore, but rather a group of multiple CPU cores
that can operate and compute independent of each other.

Unfortunately mainstream languages are still struggling with delivering
language concepts and \textit{semantics} that allow for utilizing this new
hardware. In Fig.\ref{pythreads} one can see a very minimal multithreaded
program written in Python, one of the most used languages today
\cite{langrank}.


\begin{wrapfigure}{l}{0.55\textwidth}
    \begin{lstlisting}
from threading import Thread

def count(n):
  while n > 0:
    n -= 1

t1 = Thread(target=count,args=(1000000,))
t1.start()

t2 = Thread(target=count,args=(1000000,))
t2.start()

t1.join()
t2.join()
    \end{lstlisting}
  \caption{Python multithreading example}
  \label{pythreads}
\end{wrapfigure}

As was shown by Beazley in \cite{pygil} this multithreaded version is up to
two times \textit{slower} than the single threaded version and the
reason for this is the so called \textit{Global Interpreter Lock}
(GIL) used by the Python interpreter which basically prevents multiple
native threads from truly executing in parallel.

This is supposed to show how slow adoption and adjustment in the mainstream
programming language market is happening, even for a problem domain that,
as I would like to argue, is not even the most recent set of problems
programmers today need to deal with.

This shows how important it is to reevaluate exisiting programming
languages, their implementations and their methodologies whenever the
underlying machine, system or problem domain changes dramatically.
\newline

Now one could argue that Python might have been a bad example and
that other mainstream languages offer better adoption of
multithreaded programming. Fig \ref{thread-examples}
shows another two examples of how threading is expressed in programming today,
namely using the programming languages \textit{Java} \cite{java} and
\textit{Go} \cite{golang}. Altough Java and its virtual runtime environment, the
\textit{Java Virtual Machine} (JVM) offer the capability of true
concurrent execution of threads,  threads still only exist as a library
feature, not as a language primitive. They are created by overwriting
specific methods that are inherited from either another class or an
interface and are used like any other \textit{object}
in Java, an object oriented programming language.

This is supposed to represent the approach that
has been favoured in programming language design of how to deal with
a changing environment. Every new or extending aspect is hidden behind
the \textit{function call}, a core feature of the language. The result is
that the language itself and its semantics virtually never change, and
the true meaning of a plain function call is expressed on an external
napkin so to speak. This eliminates all possibilities for the compiler
or any other tooling to aid in the development process because to the
compiler the function call that is supposed to start a new thread
looks exactly the same as any other function call.
\newline

Example \ref{thread-examples}b shows a newer language called \textit{Go}.
In Go threads are called \textit{go routines} because their semantics
are defined within the language itself and are created by the keyword
\textit{go}. On the one hand this allows
any compiler or other tooling to automatically check, at compile time,
whether certain rules of behavior are implemented correctly by the
programmer but on the other hand it also delivers guarantees of said
behavior to the programmer because these go routines will behave the
same independent of the execution environment in which this code is run.
Said behavior could vary when threads are only implemented as a library.

\begin{figure}[h]
    \begin{subfigure}[b]{0.55\textwidth}

    \begin{lstlisting}
class Demo extends Thread {

  public void run(){
    System.out.println("foo");
  }

  public static void main(String args[]) {
     Demo obj = new Demo();
     obj.start();
  }
}
    \end{lstlisting}

        \caption{Java threading example}
        \label{fig:java-thread}
    \end{subfigure}
    \hfill
    \begin{subfigure}[b]{0.4\textwidth}

    \begin{lstlisting}
package main
import "fmt"

func f(from string) {
    for i := 0; i < 3; i++ {
        fmt.Println(from, ":", i)
    }
}

func main() {
  go f("goroutine")
}

    \end{lstlisting}

        \caption{Go threading example}
        \label{fig:go-thread}
    \end{subfigure}

  \caption{Threading examples in Java and Go}
  \label{thread-examples}

\end{figure}

This shows that language designs do react to dramatic change of the
underlying machine model. They include important aspects of the
execution environment and hardware as core features of the language
in order to provide a programming environment sweetspot: as close
to the hardware to be still implemented efficiently but abstract
enough so programming becomes intuitive, productive and less error prone.
Just like the Turing machine hit that sweetspot on the theoretical side.
\newline

However, compared to the switch from single core to multi core CPU systems,
the more dramatic paradigm shift in my view was the advent of
the \textit{internet}, or large scale networking in general, because it
dawned what is now known as the \textit{information age}. Although this
is a rather broad term, describing global change in almost all areas of
modern society, I believe these changes can also be seen in the area
of computing and computation. Because in order to deliver new products
like: the world wide web, internet search, internet advertising,
social networks, social media, music streaming,
video streaming, messenger services, navigation/maps, e-commerce etc.
companies had to build massive computer networks in order to either store
massive amounts of data or to simply scale their
product to millions or billions of users
and although almost all of the current products, apps, or services
could not exist without the computation that is implemented inside
their core parts, it is \textit{communication} that has become the
main obstacle for building large scale \textit{information systems}.
