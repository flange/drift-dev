\section{Summary}
\label{summary}

In summary this thesis proposed a novel programming environment
in the area of distributed programming, called the
\textit{Drift} programming environment. The environment consists
of $i)$ the Drift language, a compact and abstract distributed
coordination language focussing on the concepts of
\textit{data} and \textit{functionality} and emphasising
the importance of \textit{names} for conveying
meaning in programming $ii)$ the Drift shell an implementation
of the Drift language in form of an interpreter,
$iii)$ the Drift distributed execution
environment including the Drift FS, a novel distributed and
immutable file system based on distributed message queues and
$iv)$ the Drift user interface, a web interface featuring
the Drift shell and a graphical representation of the
Drift execution environment based on the well-known
Petri Net syntax.
\newline

To set the context for the work presented here,
chapter \ref{distributedprogramming} first introduced the concept of
\textit{distributed programming} by giving a short summary of the
history and evolution of programming languages and their adaption to
fundamental changes of the underlying hardware. This was done by first
looking at the adaption of programming language constructs regarding
the change from single-core to multi-core CPUs in chapter
\ref{LanguageEvolution}
and continued with looking at another adaption, namely from single
machine computing to distributed computing using multiple machines
connected via a network. Therefore chapter \ref{DistributedObjects}
introduced the approach of \textit{distributed objects} and
chapter \ref{actorModel} introduced the \textit{actor model},
using examples to highlight their advantages but also their shortcomings.
\newline

In order to deal with these shortcomings and to finally introduce
the Drift project, chapter \ref{relatedwork} first introduced
different software projects and people that
created these projects as well as their ideas and philosophies
regarding software, system design and system architecture. Although these
projects might seem unrelated at first, they all contributed
to the ideas and design decisions of the overall Drift project.
Therefore, for each project, the relevant aspects regarding
their influence on the Drift project were introduced and highlighted.

This was done by first introducing Rich Hickey and his ideas of
the \textit{language of the system} and \textit{value-oriented
programming} in chapter \ref{LanguageOfTheSystem}.
These ideas are then picked up in chapter \ref{cuneiform},
by introducing \textit{Cuneiform}, a functional scientific workflow language.
Cuneiform is developed here at the \textit{Humboldt-Universität zu Berlin}
and combines the ideas of immutable data from the real of
functional programming with the bird's-eye perspective of describing
a system from the scientific workflow community.

Chapter \ref{bash} then took a closer look at history and introduced
the UNIX time sharing system including its command line shell.
Given the new concepts of processes, a UNIX system can be seen as
a system of independent processes working concurrently. Therefore
the UNIX shell becomes a coordination language, orchestrating
these processes and their data flow, much like a workflow language.
Chapter \ref{bash} then went on by introducing a mapping between
the language concepts the UNIX shell and language concepts known
from functional programming languages like Cuneiform, showing
an almost one-to-one correspondence except for the global shared mutable
state which is the UNIX file system.

Chapters \ref{datomic} and \ref{git} then introduced two
rather new projects, namely \textit{Datomic} and \textit{Git}
which both deal with the aspect of \textit{immutable data}.
Datomic, being a distributed and immutable facts log, presents
its user the illusion of immutability though being implemented
on top of already existing mutable distributed storage solutions
like key-value stores or SQL data bases. Git on the other hand
is implemented as an immutable content-addressable file system
which presents its user with a mutable and stateful interface
based on ordinary files and file access.
Considering these two opposing projects challenges the widely-accepted
believe of an insurmountable divide between both approaches:
mutability and statefulness versus immutability and functional programming.

Lastly, chapter \ref{bretvictor} introduced Bret Victor and his
idea of \textit{immediate feedback}, which, in the field of programming,
can be seen as the goal of creating a tight feedback loop between
the programmer typing code or issueing commands and her being presented
with the system's response. This ties with the concept of the shell as
introduced in chapter \ref{bash} which immediately runs its user's
commands, as opposed to the more traditional approach of first
writing all of the code, compiling it and then running it.
\newline

Chapter \ref{drift} then presented the Drift project
by first focussing on communication as the essence of distributed
programming and highlighting the importance of \textit{names}
for conveying the meaning and semantics of a program in chapter
\ref{importanceofnames}. Then, the imperative and stateful Drift
coordination language was presented
in chapter \ref{driftlang}, using multiple code examples to
gradually introduce the language concepts based around names,
namespaces and services. Alongside the language itself, the shell
was introduced, which features some of the language keywords as
shell built-ins as well as letting the user traverse the names
and namespaces like he would a typical UNIX file system.

Then the Drift back end implementation was presented including
the Drift file system, which is done by gradually building up the
system from the basic communication between shell and system up to
how different message queues are used for either data and signaling
and how the file system that is presented to the user was implemented
using these message queues, especially given the mapping of directories
to namespaces. Furthermore, the error model and fault tolerance
mechanisms used by the Drift system were presented, using a control
token mechanism as well as already established distributed recovery software.
\newline

Lastly the Drift UI was presented, which was implemented in the
form of a web interface for modern browsers. It featured a full
fledged Drift shell as well as a graphical representation of
the system orchestrated by the user. Instead of a direct acyclic
graph (DAG) which is often used by similar system, the also
widely-known Petri Net syntax with an alternative semantics
was used. This allowed to naturally represent the Drift language
concepts of \textit{names} and \textit{services} as places
and transitions in the Petri Net syntax. Since this
graphical representation is not only updated by commands of
the user, but also by events in the system, e.g. finishing
of a task, it is possible that elements are removed from the
graph without user intervention. In order to allow the user
to recap and reconstruct earlier system states a time bar was
presented that presents all the events in the system and which
allows the user to print older system graphs by simply mouse-hovering
over the event entry in the time bar.
\newline

All the components presented in chapter \ref{drift} as a whole
represent the \textit{Drift Programming Environment}. A distributed
orchestration system focused on coordinating the communication between
its services, built upon the idea of distributed immutable data
which allows for virtually no contention or coordination overhead
and therefore no global coordinator.

Instead of sticking to this immutability even up to the user interface,
forcing the user into the paradigm of functional programming, an
imperative, stateful language is used, built on top of that immutable
system, that allows the user to stay in the realm of imperative
programming if that feels more natural. To make dealing with this
introduced state more managable, a graphical representation based
on Petri Net syntax is used, that allows the user to immediately
observe its actions and data but also allows to revisit earlier
system states.
\newline

Any component of the \textit{Drift Programming Environment} is
a full fledged research topic in and by itself. For example,
language design offers many aspects that have \textit{not} been considered
in the work presented here as will be discussed in chapter \ref{futurework},
but also the system implementation offers infinite possibilities
for improvement in areas like scheduling, load balancing or
garbage collection, just to name a few.
\newline

However, the main contribution of the work that was presented here
is not seen as the quality of each individual component. All of
the implemented components should rather be seen as prototypes and
proof-of-concepts that explore novel ideas in their individual
areas. The main contribution of the \textit{Drift Programming Environment}
is the collection and integration of all of these prototype
components into one large system and environment, merging
and composing them in a way that they as a collection fit
the overall goal of this work and represent more than just
their overall sum, challenging the widely-accepted believe
that there is a fundamental difference between imperative,
e.g. stateful languages and functional, e.g. immutable languages,
as has been \textit{the} dominating discussion for the last
decades in the area of programming language design.

